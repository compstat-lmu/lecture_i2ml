\documentclass[11pt,compress,t,notes=noshow, xcolor=table]{beamer}
\usepackage[]{graphicx}\usepackage[]{color}
% maxwidth is the original width if it is less than linewidth
% otherwise use linewidth (to make sure the graphics do not exceed the margin)
\makeatletter
\def\maxwidth{ %
  \ifdim\Gin@nat@width>\linewidth
    \linewidth
  \else
    \Gin@nat@width
  \fi
}
\makeatother

\definecolor{fgcolor}{rgb}{0.345, 0.345, 0.345}
\newcommand{\hlnum}[1]{\textcolor[rgb]{0.686,0.059,0.569}{#1}}%
\newcommand{\hlstr}[1]{\textcolor[rgb]{0.192,0.494,0.8}{#1}}%
\newcommand{\hlcom}[1]{\textcolor[rgb]{0.678,0.584,0.686}{\textit{#1}}}%
\newcommand{\hlopt}[1]{\textcolor[rgb]{0,0,0}{#1}}%
\newcommand{\hlstd}[1]{\textcolor[rgb]{0.345,0.345,0.345}{#1}}%
\newcommand{\hlkwa}[1]{\textcolor[rgb]{0.161,0.373,0.58}{\textbf{#1}}}%
\newcommand{\hlkwb}[1]{\textcolor[rgb]{0.69,0.353,0.396}{#1}}%
\newcommand{\hlkwc}[1]{\textcolor[rgb]{0.333,0.667,0.333}{#1}}%
\newcommand{\hlkwd}[1]{\textcolor[rgb]{0.737,0.353,0.396}{\textbf{#1}}}%
\let\hlipl\hlkwb

\usepackage{framed}
\makeatletter
\newenvironment{kframe}{%
 \def\at@end@of@kframe{}%
 \ifinner\ifhmode%
  \def\at@end@of@kframe{\end{minipage}}%
  \begin{minipage}{\columnwidth}%
 \fi\fi%
 \def\FrameCommand##1{\hskip\@totalleftmargin \hskip-\fboxsep
 \colorbox{shadecolor}{##1}\hskip-\fboxsep
     % There is no \\@totalrightmargin, so:
     \hskip-\linewidth \hskip-\@totalleftmargin \hskip\columnwidth}%
 \MakeFramed {\advance\hsize-\width
   \@totalleftmargin\z@ \linewidth\hsize
   \@setminipage}}%
 {\par\unskip\endMakeFramed%
 \at@end@of@kframe}
\makeatother

\definecolor{shadecolor}{rgb}{.97, .97, .97}
\definecolor{messagecolor}{rgb}{0, 0, 0}
\definecolor{warningcolor}{rgb}{1, 0, 1}
\definecolor{errorcolor}{rgb}{1, 0, 0}
\newenvironment{knitrout}{}{} % an empty environment to be redefined in TeX

\usepackage{alltt}
\newcommand{\SweaveOpts}[1]{}  % do not interfere with LaTeX
\newcommand{\SweaveInput}[1]{} % because they are not real TeX commands
\newcommand{\Sexpr}[1]{}       % will only be parsed by R



\usepackage[english]{babel}
\usepackage[utf8]{inputenc}

\usepackage{dsfont}
\usepackage{verbatim}
\usepackage{amsmath}
\usepackage{amsfonts}
\usepackage{bm}
\usepackage{csquotes}
\usepackage{multirow}
\usepackage{longtable}
\usepackage{booktabs}
\usepackage{enumerate}
\usepackage[absolute,overlay]{textpos}
\usepackage{psfrag}
\usepackage{algorithm}
\usepackage{algpseudocode}
\usepackage{eqnarray}
\usepackage{arydshln}
\usepackage{tabularx}
\usepackage{placeins}
\usepackage{tikz}
\usepackage{setspace}
\usepackage{colortbl}
\usepackage{mathtools}
\usepackage{wrapfig}
\usepackage{bm}
\usetikzlibrary{shapes,arrows,automata,positioning,calc,chains,trees, shadows}
\tikzset{
  %Define standard arrow tip
  >=stealth',
  %Define style for boxes
  punkt/.style={
    rectangle,
    rounded corners,
    draw=black, very thick,
    text width=6.5em,
    minimum height=2em,
    text centered},
  % Define arrow style
  pil/.style={
    ->,
    thick,
    shorten <=2pt,
    shorten >=2pt,}
}
\usepackage{subfig}


% Defines macros and environments
\input{../../style/common.tex}

%\usetheme{lmu-lecture}
\newcommand{\titlefigure}{figure_man/intro-titlefig.jpg}
\newcommand{\learninggoals}{
\item Understand the goal of performance estimation
\item Know the definition of generalization error
\item Understand the difference between outer and inner loss}
\usepackage{../../style/lmu-lecture}

\let\code=\texttt
\let\proglang=\textsf

\setkeys{Gin}{width=0.9\textwidth}

\title{Introduction to Machine Learning}
% \author{Bernd Bischl, Christoph Molnar, Daniel Schalk, Fabian Scheipl}
\institute{\href{https://compstat-lmu.github.io/lecture_i2ml/}{compstat-lmu.github.io/lecture\_i2ml}}
\date{}

\setbeamertemplate{frametitle}{\expandafter\uppercase\expandafter\insertframetitle}


\begin{document}


% This file loads R packages, configures knitr options and sets preamble.Rnw as parent file
% IF YOU MODIFY THIS, PLZ ALSO MODIFY setup.Rmd ACCORDINGLY...


% Defines macros and environments
\input{../../latex-math/basic-math.tex}
\input{../../latex-math/basic-ml.tex}
\input{../../latex-math/ml-automl.tex}
%! includes: basics-learners 

\lecturechapter{Evaluation: Introduction and Remarks}
\lecture{Introduction to Machine Learning}

% ------------------------------------------------------------------------------

\begin{vbframe}{Example Practical Method}
Given: 20 training observations, 12 negative and 8 positive

\vspace{20pt}

\tiny
\begin{tabular}{l|p{0.1cm}|p{0.1cm}|p{0.1cm}|p{0.1cm}|p{0.1cm}|p{0.1cm}|p{0.1cm}|p{0.1cm}|p{0.1cm}|p{0.1cm}|p{0.1cm}|p{0.1cm}|p{0.1cm}|p{0.1cm}|p{0.1cm}|p{0.1cm}|p{0.1cm}|p{0.1cm}|p{0.1cm}|p{0.1cm}}

  \hspace{-8pt} \#&  1& 2&  3&  4&  5&  6&  7&  8&  9& 10& 11& 12& 13& 14& 15& 16& 17& 18& 19& 20 \\ \hline
  \hspace{-8pt} C & N & N & N & N & N & N & N & N & N & N & N & N & P & P & P & P & P & P & P & P \\ \hline
  \hspace{-8pt} Score & .18 & .24 &  .32 & .33 & .4 & .53 & .58 & .59 & .6 & .7 & .75 & .85 & .52 & .72 & .73 & .79 & .82 & .88 & .9 &.92
\end{tabular}
\normalsize

\vspace{20pt}
$\Rightarrow$ sort by score and draw the curves:
\vspace{20pt}

\tiny
\begin{tabular}{l|p{0.1cm}|p{0.1cm}|p{0.1cm}|p{0.1cm}|p{0.1cm}|p{0.1cm}|p{0.1cm}|p{0.1cm}|p{0.1cm}|p{0.1cm}|p{0.1cm}|p{0.1cm}|p{0.1cm}|p{0.1cm}|p{0.1cm}|p{0.1cm}|p{0.1cm}|p{0.1cm}|p{0.1cm}|p{0.1cm}}

  \hspace{-8pt} \#&  20& 19&  18&  12&  17&  16&  11&  15&  14& 10& 9& 8& 7& 6& 13& 5& 4& 3& 2& 1 \\ \hline
  \hspace{-8pt} C & P & P & P & N & P & P & N & P & P & N & N & N & N & N & P & N & N & N & N & N \\ \hline
  \hspace{-8pt} Score & .92 & .9 &  .88 & .85 & .82 & .79 & .75 & .73 & .72 & .7 & .6 & .59 & .58 & .53 & .52 & .4 & .33 & .32 & .24 &.18
\end{tabular}
\normalsize
\end{vbframe}


\begin{vbframe}{Example Practical Method}
\begin{center}
\includegraphics[width=0.75\textwidth]{figure_man/roc-curve-ex2.png}
\end{center}
\begin{itemize}
  \item Best accuracy achieved with observation \# 18.
  \item Setting $\theta = 0.88 \Rightarrow$ accuracy of $15/20 \; \hat{=} \; 75 \%$.
\end{itemize}
\end{vbframe}



\begin{vbframe}{Explanation Mann-Whitney-U Test}
\begin{itemize}
\item First we plot the ranks of all the scores as a stack of horizontal bars, and color them by the labels.
\item Stack the green bars on top of one another, and slide them horizontally as needed to get a nice even stairstep on the right edge (See: practical method example for ROC curves):
\end{itemize}
\begin{center}
\includegraphics[width=0.8\textwidth]{figure_man/roc-mannwhitney3.png}
\end{center}


\framebreak


\begin{center}
\includegraphics[width=0.5\textwidth]{figure_man/roc-mannwhitney2.png}
\end{center}

\begin{itemize}
 \item Definition of the U statistic: $U = R_1 - \cfrac{n_1(n_1 + 1)}{2}$
 \begin{itemize}
  \item $R_1$ is the sum of ranks of positive cases (the area of the green bars)
  \item $n_1$ is the number of positive cases
 \end{itemize}
  \item The area of the green bars on the right side is equal to $\cfrac{n_1(n_1 + 1)}{2}$.
\end{itemize}

\framebreak
\begin{center}
\includegraphics[width=0.5\textwidth]{figure_man/roc-mannwhitney2.png}
\end{center}

\begin{itemize}
 \item $U =$ area of the green bars on left side
 \item area of dashed rectangle = $n_1 \cdot n_2$
 \item $AUC$ is $U$ normalized to the unit square,
\end{itemize}
$$\Longrightarrow AUC = \cfrac{U}{n_1\cdot n_2}$$
with $n_1 = \text{POS}$ and $n_2 = \text{NEG}$.
\end{vbframe}


\begin{vbframe}{Partial AUC}
\begin{itemize}
  \item Sometimes it can be useful to look at a \href{http://journals.sagepub.com/doi/pdf/10.1177/0272989X8900900307}{specific region under the ROC curve}  $\Rightarrow$ partial AUC (pAUC).
  \item Let $0 \leq c_1 < c_2 \leq 1$ define a region.
  \item For example, one could focus on a region with low fpr ($c_1 = 0, c_2 = 0.2$) or a region with high tpr ($c_1 = 0.8, c_2 = 1$):
\end{itemize}

\begin{center}
\includegraphics[width=0.6\textwidth]{figure/partial-roc-1.pdf}
\end{center}

% <<echo = FALSE, message = FALSE, warning = FALSE, fig.width = 14, fig.height = 7, out.width="0.7\\textwidth">>=
% library(pROC)
% set.seed(1)
% D.ex <- rbinom(200, size = 1, prob = .5)
% M1 <- rnorm(200, mean = D.ex, sd = .65)
% M2 <- rnorm(200, mean = D.ex, sd = 1.5)
% 
% test <- data.frame(D = D.ex, D.str = c("Healthy", "Ill")[D.ex + 1],
%                    M1 = M1, M2 = M2, stringsAsFactors = FALSE)
% 
% rocobj <- pROC::roc(test$D, test$M1)
% par(mfrow = c(1, 2))
% pROC::plot.roc(rocobj, print.auc=TRUE, auc.polygon=TRUE, partial.auc=c(1, 0.8), partial.auc.focus="sp", reuse.auc=FALSE, legacy.axes = TRUE, xlab = "fpr", ylab = "tpr", xlim = c(1, 0), ylim = c(0, 1),  auc.polygon.col="red", auc.polygon.density = 20, auc.polygon.angle = 135, partial.auc.correct = FALSE
%  )
% pROC::plot.roc(rocobj, print.auc=TRUE, auc.polygon=TRUE, partial.auc=c(1, 0.8), partial.auc.focus="se", reuse.auc=FALSE, legacy.axes = TRUE, xlab = "fpr", ylab = "tpr", xlim = c(1, 0), ylim = c(0, 1),  auc.polygon.col="red", auc.polygon.density = 20, auc.polygon.angle = 135)
% @


\framebreak

\begin{itemize}
  \item $\text{pAUC} \in [0, c_2 - c_1]$.
  \item The partial AUC can be corrected (see \href{http://journals.sagepub.com/doi/pdf/10.1177/0272989X8900900307}{McClish}), to have values between $0$ and $1$, where $0.5$ is non discriminant and $1$ is maximal: $$\text{pAUC}_\text{corrected} = \cfrac{1+\cfrac{\text{pAUC} - \text{min}}{\text{max} - \text{min}}}{2} $$
  \item $\text{min}$ is the
value of the non-discriminant AUC in the region
  \item $\text{max}$ is the maximum possible AUC in the region
\end{itemize}
\end{vbframe}



\begin{vbframe}{Multiclass AUC}
\begin{itemize}
  \item Consider multiclass classification, where a classifier predicts the probability $p_k$ of belonging to class $k$ for each class.
  \item Hand and Till (2001) proposed to average the AUC of pairwise comparisons (1 vs. 1) of a multiclass classifier.
  \begin{itemize}
    \item estimate $AUC(i,j)$ for each pair of class $i$ and $j$
    \item $AUC(i,j)$ is the probability that a randomly drawn member of class $i$ has a lower probability of belonging to class $j$
      than a randomly drawn member of class $j$.
    \item for $K$ classes, we have ${{K}\choose{2}} = \tfrac{K (K-1)}{2}$ values of $AUC(i,j)$ that are then averaged to compute the Multiclass AUC.
  \end{itemize}
\end{itemize}
\end{vbframe}

\begin{vbframe}{Calibration and Discrimination}
We consider data with a binary outcome $y$.
\begin{itemize}
  \item \textbf{Calibration:} When the predicted probabilities closely agree
    with the observed outcome (for any reasonable grouping).
  \begin{itemize}
    \item \textbf{Calibration in the large} is a property of the \textit{full sample}.
    It compares the observed probability in the full sample  (e.g. proportion of observations for which $y=1$)
   % <!-- (e.g., 10% if 10 of 100 individuals have the outcome being predicted, e.g. $y=1$) -->
    with the average predicted probability in the full sample.
    \item \textbf{Calibration in the small} is a property of \textit{subsets} of the sample.
    It compares the observed probability in each subset with the average
    predicted probability in that subset.
  \end{itemize}
  \item \textbf{Discrimination:} Ability to perfectly separate the population into $y=0$ and $y=1$.
    Measures of discrimination are, for example, AUC, sensitivity, specificity.
\end{itemize}
\end{vbframe}

\begin{vbframe}{Calibration and Discrimination}
%<!-- http://www.uphs.upenn.edu/dgimhsr/documents/predictionrules.sp12.pdf -->
A well calibrated  classifier can be poorly discriminating, e.g.

\begin{table}[]
\centering
\begin{tabular}{rrrr}
\hline
Obs. Nr. & truth & Pred Rule 1 & Pred Rule 2 \\
\hline
1        & 1     & 1           & 0           \\
2        & 1     & 1           & 0           \\
3        & 0     & 0           & 1           \\
4        & 0     & 0           & 1           \\ \hline
Avg Prob & 50\%  & 50\%        & 50\%        \\
\hline
\end{tabular}
\end{table}

\begin{itemize}
  \item Both prediction rules have identical calibration in the large (50\%), however, rule 1 is better than rule 2.
\end{itemize}

% <<eval = FALSE, echo = FALSE>>=
% truth = c(1,1,0,0,0,0)
% pred.rule.1 = c(1,1,0,0,0,0)
% pred.rule.2 = c(0,0,0,0,1,1)
% kable(data.frame(truth = truth, "pred rule 1" = pred.rule.1, "pred rule 2" = pred.rule.2))
% @
\end{vbframe}

\begin{vbframe}{Calibration and Discrimination}
A well discriminating classifier can have a bad calibration, e.g.

\begin{table}[]
\centering
\begin{tabular}{rrrr}
\hline
Obs. Nr. & truth & Pred Rule 1 & Pred Rule 2 \\
\hline
1        & 1     & 0.9           & 0.9         \\
2        & 1     & 0.9           & 0.9           \\
3        & 0     & 0.1          & 0.7           \\
4        & 0     & 0.1         & 0.7           \\ \hline
Avg Prob & 50\%  & 50\%        & 80\%        \\
\hline
\end{tabular}
\end{table}

\begin{itemize}
  \item Both prediction rules are well discriminating (e.g., setting thresholds $\theta_1 = 0.5$, $\theta_2 = 0.8$)
  \item Prediction rule 2 is rather poorly calibrated. The proportion of observations for which $y=1$ would be estimated with $80\%$.
\end{itemize}
\end{vbframe}

\begin{vbframe}{ROC Analysis in R}
\begin{itemize}
  \item \texttt{generateThreshVsPerfData} calculates one or several performance measures for a sequence of decision thresholds from 0 to 1.
  \item It provides S3 methods for objects of class \texttt{Prediction}, \texttt{ResampleResult}
and \texttt{BenchmarkResult} (resulting from  \texttt{predict.WrappedModel}, \texttt{resample}
or \texttt{benchmark}).
  \item \texttt{plotROCCurves} plots the result of \texttt{generateThreshVsPerfData} using \texttt{ggplot2}.
  \item More infos \url{http://mlr-org.github.io/mlr-tutorial/release/html/roc_analysis/index.html}
\end{itemize}
\end{vbframe}

\begin{vbframe}{Example 1: Single predictions}

\textcolor{blue}{small code chunk}
% \scriptsize
% <<echo=TRUE, message = FALSE>>=
% library(mlr)
% set.seed(1)
% # get train and test indices
% n = getTaskSize(sonar.task)
% train.set = sample(n, size = round(2/3 * n))
% test.set = setdiff(seq_len(n), train.set)
% 
% # fit and predict
% lrn = makeLearner("classif.lda", predict.type = "prob")
% mod = train(lrn, sonar.task, subset = train.set)
% pred = predict(mod, task = sonar.task, subset = test.set)
% @
\normalsize
\end{vbframe}

\begin{vbframe}{Example 1: Single predictions}
We calculate fpr, tpr and compute error rates:

\scriptsize
\textcolor{blue}{one line of code}
% <<echo = TRUE>>=
% df = generateThreshVsPerfData(pred, measures = list(fpr, tpr, mmce))
% @
\normalsize
\begin{itemize}
  \item \texttt{generateThreshVsPerfData} returns an object of class \texttt{ThreshVsPerfData},
which contains the performance values in the \texttt{\$data} slot.
  \item By default, \texttt{plotROCCurves} plots the performance values of the first two measures passed
to \texttt{generateThreshVsPerfData}.
  \item The first is shown on the x-axis, the second on the y-axis.
\end{itemize}
\end{vbframe}

\begin{vbframe}{Example 1: Single predictions}
\scriptsize
\textcolor{blue}{one line of code + figure}
% <<echo = TRUE, fig.align="center", fig.width = 5, fig.height = 5, out.width="0.55\\textwidth">>=
% df = generateThreshVsPerfData(pred, measures = list(fpr, tpr, mmce))
% plotROCCurves(df)
% @
\normalsize

\framebreak

The corresponding area under curve auc can be calculated by

\scriptsize
\textcolor{blue}{one line of code}
% <<echo = TRUE>>=
% performance(pred, auc)
% @

\normalsize
\texttt{plotROCCurves} always requires a pair of performance measures that are plotted against
each other.

\framebreak

If you want to plot individual measures vs. the decision threshold, use

\scriptsize
\textcolor{blue}{one line of code + figure}
% <<echo = TRUE, fig.align="center", fig.height = 4, fig.width = 8, out.width="0.9\\textwidth">>=
% plotThreshVsPerf(df)
% @
\normalsize
\end{vbframe}


\begin{vbframe}{Example 2: Benchmark Experiment}
\scriptsize
\textcolor{blue}{small code chunk}
% <<>>=
% options(width = 200)
% @
% <<echo = TRUE>>=
% lrn1 = makeLearner("classif.randomForest", predict.type = "prob")
% lrn2 = makeLearner("classif.rpart", predict.type = "prob")
% 
% cv5 = makeResampleDesc("CV", iters = 5)
% 
% bmr = benchmark(learners = list(lrn1, lrn2), tasks = sonar.task,
%   resampling = cv5, measures = list(auc, mmce), show.info = FALSE)
% bmr
% @
\normalsize

Calling \texttt{generateThreshVsPerfData} and \texttt{plotROCCurves} on the \texttt{BenchmarkResult}
produces a plot with ROC curves for all learners in the experiment.

\framebreak


\scriptsize
\textcolor{blue}{one line of code + figure}
% <<echo = TRUE, fig.align="center", fig.height = 4, fig.width = 8, out.width="\\textwidth">>=
% df = generateThreshVsPerfData(bmr, measures = list(fpr, tpr, mmce))
% plotROCCurves(df)
% @
\framebreak

\scriptsize
\textcolor{blue}{one line of code + figure}
% <<echo = TRUE, fig.align="center", fig.height = 4, fig.width = 8, out.width="\\textwidth">>=
% plotThreshVsPerf(df)
% @
\end{vbframe}

% ------------------------------------------------------------------------------

\endlecture
\end{document}
